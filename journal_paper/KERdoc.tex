% KERdoc.tex V1.0, 14 June 2004

\documentclass{KERauth}
%\usepackage{times}

\begin{document}
\KER{1}{24}{00}{0}{2004}{S000000000000000}

\runningheads{A. Smith}{A demonstration of The Knowledge
Engineering Review class file}

%\doublespacing

\title{A demonstration of the \LaTeXe\
class file for\\ \itshape{The Knowledge Engineering
Review}\footnote{Please ensure that you have the most up to date
class file. This is available from\\
\texttt{ftp://ftp.cup.cam.ac.uk/pub/texarchive/journals/latex/}}}

\author{ALISTAIR SMITH}

\address{Sunrise Setting Ltd, 12a Fore Street, St.~Marychurch,
Torquay, Devon, TQ1~4NE, UK\\
\email{alistair.smith@sunrise-setting.co.uk}}

\begin{abstract}
This short note describes the use of the \LaTeXe\ \textsf{KERauth.cls}
class file for setting papers for \emph{The Knowledge Engineering
Review} published by Cambridge University Press.
\end{abstract}

\section{Introduction}
Many authors submitting to research journals now use \LaTeXe\ to
prepare their papers and supply a copy of their code to the
publishers. Here we describe the \textsf{KERauth.cls} class file
which can be used to convert articles produced with other \LaTeXe\
class files into a form similar to that used to publish
\emph{The Knowledge Engineering Review}.

The \textsf{KERauth.cls} class file preserves much of the standard
\LaTeXe\ interface so that any document which was produced using
the standard \LaTeXe\ \textsf{article} style can easily be
converted to work with the \textsf{KER} style. However, the width
of text and typesize may vary from that of \emph{article};
therefore \emph{line breaks will change} and it is possible that
any tabular material, computer listings and displayed mathematics
may need resetting.

In the following sections we describe how to lay out your code to
use \textsf{KERauth.cls} to reproduce the typographical look of the
\emph{Journal}. However, this note is not a guide to using
\LaTeXe\ and we would refer you to any of the many books available
(see, for example, Goossens \emph{et~al.} (1994), Kopka
\& Daly (1995) and Lamport (1994)).

In addition, you should consult the editorial office for more general
instructions about submission and preparation of your paper.

\section{The three golden rules}
Before we proceed, we would like to stress \emph{three golden rules}
that need to be followed to enable the most efficient use of your code
at the typesetting stage:
\begin{enumerate}
\item[(i)] keep your own macros to an absolute minimum;
\item[(ii)] avoid inserting extra horizontal and vertical spacing (except in
accepted cases, such as using \verb"\," before a differential~d or \verb"\quad"
to separate an equation from its qualifier);
\item[(iii)] follow \emph{The Knowledge Engineering Review} reference
style (see a recent copy, post the 2003 volume).
\end{enumerate}

\section{Getting started}
The \textsf{KERauth} class file should run on any standard \LaTeXe\
installation: simply place \textsf{KERauth.cls} in your system's usual
directory (do not forget to rebuild the ls-R database files if
required by your \TeX\ distribution!).

If any of the fonts, class files or packages it requires are
missing from your installation, they can be found on the
\emph{\TeX\ Live} CD-ROMs or from CTAN.

The \emph{Journal} is published using Times fonts, but as some
authors will not have these installed on their local \TeX\
systems, \textsf{KERauth.cls} uses Computer Modern fonts by
default. If you have Times fonts installed, you need only
uncomment the line \verb"%\usepackage{times}"
in the article header.

\section{The article header information}
The heading for any file using \textsf{KERauth.cls} is:

\begin{verbatim}
\documentclass{KERauth}
%\usepackage{times}

\begin{document}
\KER{<First page>}{<last page>}{00}{0}{2004}{S000000000000000}

\runningheads{<Initials and surname>}{<Short title>}

%\doublespacing

\title{<Minimal use of capitals>}

\author{AN AUTHOR\affilnum{1},
SOMEONE ELSE\affilnum{2} and
PERHAPS ANOTHER\affilnum{1}}

\address{\affilnum{1}First author's address
(in this example it is the same as the third author)\\
\email{First and third authors email addresses}\\
\affilnum{2}Second author's address\\
\email{second author's email address}}

\begin{abstract}
<Text>
\end{abstract}

\section{Introduction}
.
.
.
\end{verbatim}

\subsection{Remarks}
\begin{enumerate}
\item[(i)] In \verb"\runningheads", keep the short title and the
authors' details to no more than 50 characters each; use
`\emph{et~al.}' if more than two authors.

\item[(ii)]
Note the use of \verb"\affil" and \verb"\affilnum"
to link names and addresses.

\item[(iii)] Avoid explicit reference citations and displayed
mathematics in the abstract.


\item[(iv)] For submitting a double-spaced manuscript, uncomment
\verb"%\doublespacing"

\end{enumerate}

\section{The body of the article}

\subsection{Section headings}
Articles are normally divided into sections and possibly
subsections and subsubsections. The command
\verb"\section{<title>}" is used to start a section and
\verb"\subsection{<title>}" a subsection. An unnumbered section is
obtained by adding an asterisk before the opening brace.

An Acknowledgement section is started with \verb"\acks" or
\verb"\ack" for \textit{Acknowledgements} or
\textit{Acknowledgement}, respectively. They appear just before
the references and any appendices.

\subsection{Mathematics}
\textsf{KERauth.cls} makes the full functionality of \AmS\/\TeX\
available. We encourage the use of the \verb"align", \verb"gather"
and \verb"multline" environments for displayed mathematics.

\subsection{Figures}
\textsf{KERauth.cls} uses the \textsf{graphicx} package for handling
figures.  The default device driver is \textsf{dvips}. You may
need to specify a different option in the \verb"\documentclass"
line to match your system. For example,
\verb+\documentclass[dvipsone]{KERauth}+

Figures are called in as follows:
\begin{verbatim}
\begin{figure}
\centering
\includegraphics[<width=??pc>]{<figure eps name>}
\caption{<Figure caption>}
\end{figure}
\end{verbatim}

For further details on how to size figures etc. with the
\textsf{graphicx} package, see, for example, Goossens
\emph{et~al.} (1994) and Kopka \& Daly (1995). If figures are
available in an acceptable format (for example, .eps,~.ps) they
will be used but a printed version should always be provided.

\subsection{Tables}
\textsf{KERauth.cls} uses the \textsf{booktabs} package to
enhance the appearance of tables.

The standard coding for a table is:
\begin{verbatim}
\begin{table}
\caption{<Table caption>}
\begin{center}
\begin{small}
\begin{tabular}{<table alignment>}
\toprule
<column headings>\\
\midrule
<table entries (separated by & as usual)\\
<table entries>\\
\bottomrule
\end{tabular}
\end{small}
\end{center}
\end{table}
\end{verbatim}

\subsection{Cross-referencing}
The use of the \LaTeX\ cross-reference system
for figures, tables, equations, etc, is encouraged
(using \verb"\ref{<name>}" and \verb"\label{<name>}").

\subsection{References}
References should follow the Harvard (name/date) system and the
reference list at the end of the paper should be in
\emph{alphabetical order}.
For the general reference style, see a recent copy (post the 2003
volume) of the journal.

The commands for the Harvard-style reference list are:
\begin{verbatim}
\begin{thebibliography}
\item
<Your reference>

\item
<Your reference>
.
.
.
\end{thebibliography}
\end{verbatim}

\section{The last word}
That really is all you should need to know to prepare your paper
using \textsf{KERauth.cls}.
You do, of course, have the option to call in any of your
favourite packages for setting maths, graphics, tabular material,
computer listings, etc.

\subsection{Support for \textsf{KERauth.cls}}
We offer on-line support to participating authors. Please
contact us via email at\\
\texttt{alistair.smith@sunrise-setting.co.uk}\\
We would welcome any feedback, positive or otherwise, on your
experiences of using \textsf{KERauth.cls}.

\begin{thebibliography}
\item
Goossens,~M., Mittelbach,~F. \& Samarin,~A. 1994 \emph{The \LaTeX\ Companion}.
Addison-Wesley.

\item
Kopka,~H. \& Daly,~P.~W. 1995 \emph{A Guide to \LaTeXe:
Document Preparation for Beginners and Advanced Users}, 2nd~edn.
Addison-Wesley.

\item
Lamport,~L. 1994 \emph{\LaTeX: a Document Preparation System}, 2nd~edn.                             % Title (edition).
Addison-Wesley.
\end{thebibliography}

\end{document}
